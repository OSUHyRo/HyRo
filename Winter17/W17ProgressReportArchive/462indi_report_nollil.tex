\documentclass{article}

\title{CS 462 Final Individual Report}

\author{Layne Nolli Team 28}

\begin{document}
\maketitle

\section{Project Recap:}

Our project is the computer science component of the Hybrid Rocket development project. Our purposes as the CS component of the team have been to build a system, and a supporting UI, to collect, interpret, and view the data collected during the hybrid rocket launch in June. Our team doesn't get many trial and error runs, we are a supporting element on the hybrid team and we exist to help next years development team have some real data and tools to work with that data, to develop a better rocket.\\

Currently our progress towards our goal is "mostly done" our beta code is functioning, running off of simulated data, a few small portions are still needing to be finalized. The current state of the project is as follows: We can collect data from our sensors and store it in file, we then read from that file to populate the data fields that are used to visually represent the UI, during rocket flight we will repeatedly (about 4 times a second) re-draw the UI with new data being piped into the file. After the flight is over we will have all data stored and will have the means to represent it graphically for future, or closer, examination. With some luck, our project will provide a foundation for our team, and future teams to analyze and improve the hybrid rocket project going forward.\\

\section{Problems:}

The only significant problems we have encountered on the team have been ones related to communication with the ECE teams working alongside us. These issues were mainly discussions about "who was to do what" and they were mostly resolved by early-mid winter term. As is expected, our CS project requirements and their ECE requirements tend to overlap a bit and it took us a few weeks to figure out who was doing what. This was important because we both needed some legitimate work to do, while not taking the whole thing.\\

Technical problems have been minimal. All coding projects run into some "working time" hangups but none of them really stuck for too long. All coding work that I personally worked on has been rather smooth. I think the most significant hangup we experienced as a team was getting all the correct plugins up and running in visual studio for our Python Development capabilities, we needed a bunch of additional things for python as well as for our individual components like the XBee radio transciever.\\


\section{Code:}

Instead of just posting all of our code here for no reason, here are some functional snippets that I personally developed for piping data into our graphs on the UI. There are many other segments but it would blow away the length of this document to post all of it.\\

\#AdrawPlot  : Creates the graph for the Acceleration of the rocket.\\
\#           Reads from file : ASample.txt\\
class AdrawPlot:\\
        def \_\_init\_\_(self, mainwindow, x, y, w, h):\\
            self.x = x\\
            self.y = y\\
            self.window = mainwindow\\
            self.width = w\\
            self.height = h\\

            j = []\\
            k = []\\
            readFile = open('ASample.txt', 'r')\\
            sepFile = readFile.read().split('n')\\
            readFile.close()\\
            
            for plotPair in sepFile:\\
                aAndB = plotPair.split(',')\\
                j.append(int(aAndB[0]))\\
                k.append(int(aAndB[1]))\\


            f = Figure(figsize=(self.width,self.height), dpi=100)\\
            accel = f.add\_subplot(111)\\
            accel.plot(j,k)\\
            \#a.plot([1,2,3,4],[1,7,8,9])\\

            self.canvas = FigureCanvasTkAgg(f, master=mainwindow)\\
            self.canvas.show()\\

        def putScreen(self):\\
            self.canvas.get\_tk\_widget().place(x = self.x, y = self.y)\\

\section{Retrospective:}

I think the term retrospective is best outlined by my wiki-entries, also I don't remember a whole lot of specifics from 10 weeks ago.\\

\begin{center}
\begin{tabular}{ c c c }
 Positives & Deltas & Actions \\ 
 Team communication & finishing the functional code & bit of time digging in the code \\  
 Team work & getting our hardware into the rocket & ME's finalize the design, we install it \\
work mostly done & our stuff working with ECE stuff & communication \\
\end{tabular}
\end{center}


Week 1:\\
We didn't get any work opportunities over the break but once the start-of-term turmoil has settled I have full confidence in our team being able to get down to business on the project. Need to sign up for AIAA. Goals for the term are to be mostly done by the time spring rolls around to leave lots of room for last minute adjustments.\\

Week 2:\\
Set up meeting times with TA, Rocketry group meetings, and are looking to finalize a meet time for the CS branch of the Rocketry group still. Currently as a team we are waiting for the ECE teams current hardware status so we can set up some target goals and begin furthering the progress we have on the CS side of things.\\

Week 3:\\
No Class this week and we had a surprise cancellation from the TA as well. We were able to touch base with the ME/EE portions of our rocket design team during the first weekly meetup for this term however. Things seem to be going well on their end and we (the CS guys) are in a bit of a holding pattern until the ECE guys make a couple component decisions. Josh and Jason have made a little headway on the pythong GUI and I intend on tagging in for some hands-on work on that later this coming week.\\

Week 4:\\
This week we discussed design idea changes and got a good handle on what the cross disciplinary teams were up to. Josh picked up the hardware we are going to use for testing and development, so huge props to him. Development on alpha is getting going, we have a basic UI and placeholder data in place.

Week 5:\\
Learning about the hardware is going well. Josh got us caught up on the hardware he picked up for testing and we had a good opportunity during our group meeting to catch up on where the code is at and what needs to be worked on. A lot of other class work and midterms are falling due around this time so the group and the cross disciplinary groups have had their hands quite full. Going to be dealing with all of this paper revision and the presentation this wekk. Looking forward to making some more headway in the coming week on our code development.\\

Week 6:\\
This week we got a good look and some time to work on our code and on our presentation. As a team we planned the work on our documents and on the presentation then worked the plan and everything came together swimmingly. I set up the OneNote for our team and everyone had a super easy time getting comfortable with it and getting all our documents loaded up into it. This term is going well for our team.\\

Week 7:\\
Despite having both Josh and I sick for this week. We are still on track. We have work planned for next week and we have still been making our TA and group meetings. We are caught up on all due dates and things are looking good. Otherwise, didn't make a huge amount of progress this work.\\

Week 8:\\
This week I was able to get a large amount of progress made on the functional code of our progress and have about 50\% of our UI (the 5 graphs) running off of simulated data that is essentially the same as when we run it live. This means that that portion of the code is essentially done. Moving forward will be getting the gauges running properly as well. Then looking ahead we are just finalizing getting the ECE team's data to come in as a unified format so we can use it. Also going to be hitting these next few paper deadlines and the final presentation.\\

Week 9:\\
The guidelines for requirements shifted around a bit but we're a pretty solid team and should have no probelm dealing with the changes. Moving forward we already have most of Beta ready to go, with a few more functional segments of code to finish and we will be 90\% for project completion.\\

Week 10:\\
Been a crazy week, a lot of stuff to do, tests to take, and projects to submit but things are looking good. We are on track for beta and full release in spring term.\\

\section{Team member Eval:}

Layne Nolli (Myself): All round good team mate and Team coder. Did writing when needed, definitely more significant work done on the code side. Took many deep dives into the code to get things working for end-release. Took things from a static UI with static values as a "visual demostration" to a working UI running off simulated values in an external file. I also designed the initial run of our UI and how it should function under the given requirements.\\

Josh Asher: Team lead and writer. Josh took initiative to get started on a lot of new requirements and went out of his way to communicate with the ECE guys early in the term. He did a large amount of the writing early in the term as well, which was much appreciated. Additionally, Josh acquired some hardware for our team (out of pocket) so we would have it to work with and run tests with independent of the team budget. This was vital to getting some communication between devices started.\\

Jason Klindtworth: Team Designer. Jason took a lead role in helping design and develop the look of our UI as well as took the lead on the first poster draft.\\


\section{Conclusion:}

Looking forward to finishing the code and getting to launch that rocket.\\



\end{document}